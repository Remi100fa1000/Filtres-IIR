\documentclass{article}

\usepackage[utf8]{inputenc}
\usepackage[T1]{fontenc}
\usepackage[francais]{babel}

\usepackage{amsmath}
\usepackage{amssymb}

\usepackage{graphicx}

\usepackage{stmaryrd}

\begin{document}
\title{Filtres numériques à réponse impulsionnelle infinie (IIR) simples à implémenter}
\author{Remi100fa1000}

\maketitle

\section{Objectif}

On s'intéresse aux filtres à réponse impulsionnelle infinie (IIR de l'anglais Infinite Impulse Response). On cherche des filtres simples à implémenter sur FPGA.

\section{Filtres d'ordre 1}

Commençons par les filtres d'ordre 1 dont la réponse dans le domaine de Laplace s'écrit :

\begin{equation}
H(p) = \frac{1}{1+\frac{p}{2\pi f_c}}.
\end{equation}

Dans cette équation, $f_c$ est la fréquence de coupure du filtre.

Pour obtenir la représentation numérique de ce filtre, on utilise la transformée bilinéaire du plan :

\begin{equation}
p \approx 2f_e \frac{1-z^{-1}}{1+z^{-1}}.
\end{equation}

Où $f_e$ est la fréquence d'échantillonnage. On obtient alors :

\begin{equation}
H(z) = \frac{1+z^{-1}}{1+z^{-1}+\frac{f_e}{\pi f_c}(1-z^{-1})}.
\end{equation}

On a alors :

\begin{equation}
y(z)(1+\frac{f_e}{\pi f_c})+y(z)z^{-1}(1-\frac{f_e}{\pi f_c})= (1+z^{-1})x(z).
\end{equation}

On peut ensuite passer dans le domaine temporel pour avoir l'expression du filtre :

\begin{equation}
y(n)(1+\frac{f_e}{\pi f_c})=y(n-1)(\frac{f_e}{\pi f_c}-1)+x(n)+x(n-1).
\end{equation}

En choisissant $\frac{f_e}{\pi f_c}=2^{N}-1$ ($N>1$), on obtient l'expression suivante :

\begin{equation}
y(n)2^N=y(n-1)(2^N-2)+x(n)+x(n-1).
\end{equation}

On peut donc calculer la valeur de $y(n)$ avec la formule :

\begin{equation}
y(n)=y(n-1)-\frac{y(n-1)}{2^{N-1}}+\frac{(x(n)+x(n-1))}{2^N}.
\end{equation}

Ce filtre est simple à calculer sur FPGA car il n'utilise que des divisions par des puissances de $2$ (décalage à droite des données binaires).

\section{Filtres d'ordre 2}

\subsection{Développements préliminaires}

On cherche maintenant à trouver l'expression de filtres d'ordre 2 qui sont eux aussi simples à implémenter.

L'expression générale d'un filtre d'ordre 2 de pulsation propre $\omega_0$ a pour expression dans le domaine de Lapalce :

\begin{equation}
H(p) = \frac{1}{1+\frac{p}{\omega_0 Q}+\frac{p^2}{\omega_0^2}}.
\end{equation}

En utilisant la transformée bilinéaire du plan, on se retrouve avec l'expression :

\begin{equation}
H(z) = \frac{(1+z^{-1})^2}{\left(1+\frac{\alpha}{Q}+\alpha^2 \right)+\left(2-2\alpha^2 \right)z^{-1}+\left(1+\frac{\alpha}{Q}-\alpha^2 \right)z^{-2}},
\end{equation}

dans cette équation $\alpha=\frac{2f_e}{\omega_0}=\frac{f_e}{\pi f_c}$.

On peut alors obtenir l'expression du filtre en z dans le domaine temporel :

\begin{multline}
y(n)\left(1+\frac{\alpha}{Q}+\alpha^2 \right)+2y(n-1)\left(1-\alpha^2 \right)+y(n-2)\left(1+\frac{\alpha}{Q}-\alpha^2 \right) \\
 = x(n)-x(n-2).
\end{multline}

On a trois coefficients dans ce filtre :

\begin{equation}
1+\frac{\alpha}{Q}+\alpha^2,
\end{equation}
\begin{equation}
1-\alpha^2,
\end{equation}
\begin{equation}
1-\frac{\alpha}{Q}+\alpha^2.
\end{equation}

\subsection{Un cas particulier}

On remarque assez rapidement le cas où $Q=\frac{1}{2}$, dans ce cas là, les équations précédentes deviennent :

\begin{equation}
(1+\alpha)^2,
\end{equation}
\begin{equation}
(1+\alpha)(1-\alpha),
\end{equation}
\begin{equation}
(1-\alpha)^2.
\end{equation}

En choisissant $\alpha=2^N-1$, on obtient les coefficients : $2^{2N}$, $2^{N+1}-2^{2N}$ et $2^{2N}-2^{N+2}+4$.

Dans ce cas là, on obtient un filtre dont les coefficients s'écrivent simplement comme des somme de puissances de 2 : 

\begin{multline}
y(n)2^{2N}+2y(n-1)(2^{N+1}-2^{2N})+y(n-2)(2^{2N}-2^{N+2}+4)
\\=x(n)+2x(n-1)+x(n-2).
\end{multline}

On a ainsi un filtre IIR d'ordre 2 dont l'implémentation numérique utilise peu de ressources.

On peut noter que la fonction de transfert de ce filtre est exactement le carré de la fonction de transfert du filtre d'ordre 1 présenté dans la section précédente. En effet, cette fonction de transfert peut s'écrire :

\begin{equation}
H(z) = \left[ \frac{(1+z^{-1})}{(2^{N})+z^{-1}(2-2^{N})} \right]^2. 
\end{equation}

\subsection{Généralisation du cas particulier à l'ordre N}

On peut obtenir un filtre d'ordre N relativement simple à implémenter à partir des résultats obtenus précédemment. Bien entendu, les ressources requises pour ces filtres sont plus importantes que pour les filtres d'ordre 1 et 2 explicités ci-dessus.

Le filtre de fonction transfert 

\begin{equation}
H(z) = \left[ \frac{(1+z^{-1})}{(2^{N})+z^{-1}(1-\alpha)} \right]^K, 
\end{equation}

est un filtre d'ordre $K$ dont l'implémentation ne requiert que des multiplications par des entiers (coefficients binomiaux ou produit de coefficients binomiaux et de puissance de $2$) et une division finale par une puissance de $2$.

\subsection{Cas plus général pour $N=2$}

Le cas particulier présenté ci-dessus ne permet de réaliser qu'un nombre limité de filtres.

Pour qu'un filtre d'ordre $2$ soit simple à implémenter il suffit que:

\begin{itemize}
\item tous les coefficients du filtre soit des entiers;
\item le terme de degré zero du dénominateur soit égal à $2^N$.
\end{itemize}

La deuxième condition s'écrit :

\begin{equation}
Q = \frac{\alpha}{2^N - \alpha^2-1}.
\end{equation}

Tout d'abord, avec $Q$ sous cette forme

\begin{equation}
1-\frac{\alpha}{Q}+\alpha^2=2+2\alpha^2-2^N.
\end{equation}

Cette dernière équation nous permet de conclure que si $\alpha^2$ est entier, tous les coefficients du filtre seront entiers. 

Ensuite, une étude de la dérivée de $Q$ par rapport à $\alpha$ pour $\alpha\geq 0$ montre que cette fonction est strictement croissante sur $[0:\sqrt{2^N-1}[ \rightarrow  \mathbb{R}^+$.

Ces deux analyses nous permettent de conclure que pour construire un filtre d'odre 2 relativement simple à implémenter. Il suffit de :
\begin{itemize}
\item choisir une valeur de $N$;
\item choisir un $\alpha<\sqrt{2^N-1}$ de sorte que $\alpha^2$ soit entier;
\item calculer les coefficients du filtre avec $Q = \frac{\alpha}{2^N - \alpha^2-1}$.
\end{itemize}




\end{document}